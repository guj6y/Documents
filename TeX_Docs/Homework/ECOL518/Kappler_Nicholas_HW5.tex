\documentclass[11pt]{amsart}
\usepackage{fullpage}
\usepackage{amsmath,calc,hyperref,xcolor,tikz,pgfplots,array,tabularx,caption,pgfplotstable,epstopdf,listings,empheq}
\pgfplotsset{compat=newest}
\listfiles
\newlength\dlf
\newcommand\alignedbox[2]{
  % #1 = before alignment
  % #2 = after alignment
  &
  \begingroup
  \settowidth\dlf{$\displaystyle #1$}
  \addtolength\dlf{\fboxsep+\fboxrule}
  \hspace{-\dlf}
  \boxed{#1 #2}
  \endgroup
}
%\date{}                                           % Activate to display a given date or no date
%%\begin{tikzpicture}
%%\begin{axis}[axis lines=middle, 
%%xmin = -3, xmax = 3,
%%ymin = -1, ymax = 6,
%%thick,
%%xlabel = $x$,ylabel = $y$,
%%samples = 200,legend pos = north west,
%%ytick = {-4,2,4}, xtick = {-3,-1,1,3},x=0.7cm,y=0.6cm,
%%]
%%\addplot [very thick,domain=-3:3,red] (x,{abs(x^4-4*x^2)});
%%\legend{$h(x)$,$h(3x)$,$h(\frac{1}{3}x$}
%%\end{axis}
%%\end{tikzpicture}
 

\begin{document}
\sffamily
\section*{Ecology 518: Homework 5\\ Nick Kappler}
\subsection*{1.)} Multi-species Ricker Model*

\subsubsection*{A}
\begin{itemize}
\item [(i)] We assume the time to maturation of an egg to be a function of the total number of eggs present.  In this model, the probability of an egg maturing to a fly at a time $t$ is given by $e^{-\alpha (N_{1,x}(t)+N_{2,x}(t))}$ (one could probably derive this in terms of waiting times, or if you concentrated on the flies you could probably use the poisson distribution).  This is a form of scramble competition; the more eggs there are the harder it is for eggs to persist and mature.  Also, we don't expect eggs to persist forever, see part (ii).  
%What about eggs that don't mature & emerge at time t+1?


\item [(ii)] $s_j$ is the survival rate; can't be greater than one or less than zero since it's a rate.

\item [(iii)]  If each fly that emerges lays $R$ eggs, then the number of new eggs from a patch is given by the number of flies that emerge multiplied by $R$:
\[
RN'_{i,x}(t+1) = N_{i,x}(t+1) = RN_{i,x}(t)s_ie^{-\alpha(N_{1,x}(t) + N_{2,x}(t)}
\]
From which we easily recover $\lambda_{i,x}(t)$.

\item [(iv)]  Multi-species ricker model.

\end{itemize}

\subsubsection*{B}
Let species $i$ be the invader, species $r$ be the resident.  Let $N_{i,x}(t) = \varepsilon \ll N_{r,x}(t)$.  The equilibrium of species $r$ (in the absence of $i$) is found by calculating $\lambda_{r,x}(t) = Rs_re^{-\alpha(N_{r,x}(t))} = 1$.  This is easily solved; $N_r^* = \frac{\ln(Rs_r)}{\alpha}$.  The question is whether or not the invading species can establish itself, that is, if it has a $\lambda_{i,x}(t)$ greater than one.  Plugging in to the invading species' lifetime fitness, we get:
\begin{align*}
\lambda_{i,x} &= Rs_ie^{-\alpha(N_{r,x}^* + N_{i,x}(t))}\\
&= Rs_ie^{-\alpha(\ln(Rs_r)/\alpha + N_{i,x}(t))}\\
&= Rs_ie^{-\alpha(\ln(Rs_r)/\alpha+ \varepsilon)}\\
&= \frac{s_i}{s_r}e^{-\alpha\varepsilon}\\
\end{align*}
If we take $\lim_{\varepsilon\to0}$, we are left with $\lambda_{i,x}=\frac{s_i}{s_r}$.\footnote{For $\varepsilon\to0$, we can reasonably expect the deviation of species $r$ from equilibrium to be negligible: $N_{r,x}(t+1) =   N_{r}^*(t)Rs_re^{-\alpha(N_r^*(t)-\varepsilon)} \approx N_{r}^*Rs_re^{-\alpha N_{r}^*(t)}=N_r^*$.  Interestingly, larger $\varepsilon$ seems to only lower the growth rate of the invader, at least in the first time step.  Regardless, we must take the limit for the invasion analysis}.    This is only greater than one (i.e. $i$ can invade $r$) only if $s_i>s_r$.  Clearly, this cannot be true for both $i=1$ and $i=2$, so the two species cannot coexist.  \footnote{I thought about the cyclic/chaotic case for a while but the path forward wasn't obvious so I scrapped that effort.  I wanted to work it in terms of probability measures of chaotic orbits but I couldn't find a readily comprehensible reference on the subject.  My next thought is simply plugging in the RHS of the dynamical system of $r$, but I really want to get this off my "todo" list.}
\end{document}